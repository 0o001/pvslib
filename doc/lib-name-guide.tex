\documentclass[12pt]{article}
\usepackage{fullpage}
\usepackage{url}
\usepackage{amssymb}
\usepackage{amsmath}
\usepackage{graphics}

%\usepackage{macros}

\usepackage[dvips,usenames]{color}
%\newcommand{\ntx}{\color{red}}
\newcommand{\ntx}{}


\begin{document}

\title{Developers Guide to the NASA Libraries}

\author{Ricky Butler \\ ~\\
Assessment Technology Branch \\
Mail Stop~130 \\
NASA Langley Research Center \\
Hampton, VA~23681-2199 \\
e-mail: r.w.butler@larc.nasa.gov. 
}

\maketitle

%\begin{abstract}
%\end{abstract}


%\begin{keywords}
%\end{keywords}

%\begin{category}
%\end{category}

{\huge \sf ~\\~\\  \begin{center}

DRAFT

\end{center}
}

\newpage

%\tableofcontents


\section{Naming Conventions}

A uniform naming convention can greatly aid the prover in remembering the
names of lemmas and theorems.



\subsection{Functions: Definition and Property}

Lemmas should begin with the function name.  The key defining property
should be labeled \verb|_def|.  Although this may be just a duplication
of the body, it is convenient to have a lemma as well.  If there is a 
very common useful rewrite label it \verb|_rew|.  If there is a common
alternate or simpler version label it \verb|_lem|.
\begin{quote}
\begin{tabular}{|l|p{4.0in}|} \hline\hline
\verb|_def|   & definitional  \\ \hline
\verb|_lem|   & common simplification of alternate def \\ \hline
\verb|_rew|   & common useful rewrite: \\ \hline
\hline
\end{tabular}
\end{quote}
Typical abbreviations include:
\begin{quote}
\begin{tabular}{|l|p{4.0in}|} \hline\hline
abbrev & meaning \\ \hline
\verb|_0|   & value of function at 0 \\ \hline
\verb|_eq_0| & function equals 0: \verb|f(x) = 0 IFF ...| \\ \hline
\verb|_eq_args| & \verb|f(a,a) = ... | \\ \hline
\verb|_neg| & value of function for negated argument \verb|f(-x)| \\ \hline
\verb|_plus| & value of function for sum of arguments \verb|f(x+y)| \\ \hline
\verb|_plus1| & value of function for \verb|f(x+1)| \\ \hline
\verb|_minus| & value of function for difference of arguments \verb|f(x-y)| \\ \hline
\verb|_disj|  & disjoint\\ \hline
\verb|_dist|  & distributive \\ \hline
\verb|_comm| & commutative: \verb| f(a,b) = f(b,a)| \\ \hline
\verb|_assoc| & associative: \verb| f(a,f(b,c))) = f(f(a,b),c)| \\ \hline
\verb|_sym| & symmetry: \verb| f(-a) = f(a)| \\ \hline
\verb|_incr|   & \verb|f(a) <= f(b) IFF a <= b| \\ \hline         
\verb|_decr|   & \verb|f(a) >= f(b) IFF a <= b| \\ \hline

\verb|_strict_incr|   & \verb|f(a) < f(b) IFF a < b| \\ \hline         
\verb|_strict_decr|   & \verb|f(a) > f(b) IFF a < b| \\ \hline


\verb|_fix_pt|  & value of the defined function is a fixed point \\ \hline
\verb|_card|  & cardinality value \\ \hline

\verb|_lb|  & lower bound \\ \hline
\verb|_ub|  & upper bound \\ \hline

\verb|_lub|  & least upper bound \\ \hline
\verb|_glb|  & greatest lower bound \\ \hline

\hline

\end{tabular}
\end{quote}

\subsection{Inequalities}
\begin{quote}
\begin{tabular}{|l|p{4.0in}|} \hline\hline
\verb|_gt_0| & function gt 0: \verb| ... IMPLIES f(x) > 0 | \\ \hline
\verb|_ge_0| & function gt 0: \verb| ... IMPLIES f(x) >= 0 | \\ \hline
\verb|_lt_0| & function lt 0: \verb| ... IMPLIES f(x) < 0 | \\ \hline
\verb|_le_0| & function lt 0: \verb| ... IMPLIES f(x) <= 0 | \\ \hline
\end{tabular}
\end{quote}

\subsection{Types and Constants}
\begin{quote}
\begin{tabular}{|l|p{4.0in}|} \hline\hline
\verb|nz_|   & non zero \\ \hline
\verb|zero|  & a constant of the type which is the addition identity \\ \hline
\end{tabular}
\end{quote}



\begin{quote}
\begin{tabular}{|l|p{4.0in}|} \hline\hline
\verb|_refl| & reflexive: \verb|R(a,a)| \\ \hline
\verb|_trans| & transitive: \verb|R(x, y) & R(y, z) => R(x, z)| \\ \hline
\verb|_sym|  & symmetry property: \verb| f(-a) = f(a)| \\ \hline
\end{tabular}
\end{quote}

\subsection{Speculative}:
\begin{quote}
\begin{tabular}{|l|p{4.0in}|} \hline\hline
\verb|_diff|  & \verb|f(x) - f(y) = ...| \\ \hline
\verb|_diff_lt|  & \verb|f(x) - f(y) < ...| \\ \hline
\verb|_diff_ge|  & \verb|f(x) - f(y) >= ...| \\ \hline
\verb|_scal |  & \verb| f(a*x) = a * f(x) | \\ \hline
\verb|_pos|  & \verb| ... IMPLIES f(x) > 0 | \\ \hline
\end{tabular}
\end{quote}

\section{Examples}


\begin{verbatim}

  sqrt_def     : LEMMA sqrt(nnx) * sqrt(nnx) = nnx

  sqrt_lem     : LEMMA sqrt(nny) = nnz IFF nnz * nnz = nny

  sq_rew            : LEMMA a*a = sq(a) 

    
  sin_0        : LEMMA sin(0) = 0

  cos_plus     : LEMMA cos(a + b) = cos(a)*cos(b) - sin(a)*sin(b)

  sin_eq_0     : LEMMA sin(a) = 0 IFF EXISTS (i: int): a = i * pi
                     
  abs_diff     : LEMMA abs(x) - abs(y) <= abs(x - y)
      
  sigma_eq_arg : LEMMA sigma(low, low, F) = F(low)
      
  sq_pos       : LEMMA sq(a) >= 0

  sqrt_newton_ub(a,n)    : posreal = sqrt_newton(a,n)

\end{verbatim}

\section{Theory Names}

One theory per file and exactly the same name as the file.




%\bibliographystyle{/home/rwb/bib/NASA-srt}

\end{document}
 

